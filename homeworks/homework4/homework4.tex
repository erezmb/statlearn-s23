\documentclass{article}

\def\ParSkip{} 
\input{../../common/ryantibs}

\title{Homework 4 \\ \smallskip
\large Advanced Topics in Statistical Learning, Spring 2023 \\ \smallskip
Due Friday April 14 at 5pm}
\date{}

\begin{document}
\maketitle
\RaggedRight
\vspace{-50pt}

\section{Basic fact about CDFs and quantiles [20 pts]}

In this exercise, we'll walk through a number of basic but important facts about 
quantiles and cumulative distribution functions (CDFs). Let $F$ be a CDF, of the
form 
\[
F(x) = \P(X \leq x), \quad x \in \R,
\]
for some real-valued random variable $X$. Let $Q$ be the corresponding quantile 
function,
\[
Q(t) = \inf \{ x : F(x) \geq t \}, \quad t \in [0,1].
\]
(This is often denoted as $Q = F^{-1}$, even when the inverse of $F$ does not
exist in the usual sense.) We note that $F$ is always nondecreasing and 
right-continuous; the latter says, for any $x$,    
\[
F(x) = \lim_{y \to x^+} F(y)
\]
(where $y \to x^+$ means that $y$ approaches $x$ from the right).
Similarly, $Q$ is always nonincreasing and left-continuous; the latter says,
for any $t$,  
\[
Q(t) = \lim_{u \to t^-} Q(u)
\]
(where $u \to t^-$ means that $u$ approaches $t$ from the left).

\begin{enumerate}[label=(\alph*)]
\item Prove that for any $x$ and any $t$, 
  \marginpar{\small [3 pts]}
  \[
  F(x) \geq t \iff Q(t) \leq x.
  \]
  This is sometimes called the \emph{Galois inequality} for the quantile
  function. Hint: one direction follows from the definition of $Q$, and the
  other is a consequence of right-continuity of $F$.  

\item Use part (a) to prove that if $U \sim \mathrm{Unif}(0,1)$, then $Q(U)$ is
  distributed according to $F$ (meaning, it has $F$ as its CDF). 
  \marginpar{\small [2 pts]}

\item Use part (a) to prove that for any $t$, 
  \marginpar{\small [2 pts]}
  \[
  F(Q(t)) \geq t,
  \]
  with strict inequality if and only if $t$ is not in the range of $F$. 

\item Use parts (b) and (c) to prove that if $X$ is distributed according to
  $F$, then $F(X)$ is sub-uniform, which means that for any $t$, 
  \marginpar{\small [3 pts]}
  \[
  \P(F(X) \leq t) \leq t.
  \]
  Hint: you may start by replacing $X$ with $Q(U)$, for $U \sim
  \mathrm{Unif}(0,1)$, as they have the same distribution.  
  
\item Give an example to show that, in general, equality may fail in the result
  in part (d).  
  \marginpar{\small [2 pts]}

\item Show that we can always achieve equality in part (d) via auxiliary
  randomization: define 
  \[
  F^*(x; v) =  \lim_{y \to x^-} F(y) + v \cdot \Big( F(x) - \lim_{y \to x^-}
  F(y) \Big), 
  \]
  and prove that for $V \sim \mathrm{Unif}(0,1)$, independent of $X$, and for
  any $t$,  
  \marginpar{\small [3 pts]}
  \[
  \P(F^*(X; V) \leq t) = t.
  \]
  Hint: there are different ways to go about this; one way is to show that, with
  respect to the randomness in $(X,V)$ jointly, we can think of $F^*$ as a CDF 
  that is continuous at every point, and hence one for which we always have an
  inequality in part (c). 
\end{enumerate}

\section{Calibration-conditional beta coverage}

derive beta result do simulations

\begin{enumerate}[label=(\alph*)]
\item
  \marginpar{\small [2 pts]}
\end{enumerate}

\section{X-conditional coverage: impossible!}

\begin{enumerate}[label=(\alph*)]
\item
  \marginpar{\small [2 pts]}
\end{enumerate}


\section{}

\bibliographystyle{plainnat}
\bibliography{../../common/ryantibs}

\end{document}