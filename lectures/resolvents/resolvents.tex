\documentclass{article}

\title{Supplementary Notes: RRR (Ridge Risk Resolvents)\footnote{Any connection 
to the eponymous 2022 movie (\url{https://en.wikipedia.org/wiki/RRR_(film)}) 
is purely coincidental!\\There is neither Rise nor Roar nor Revolt here! } \\ \smallskip
\large Advanced Topics in Statistical Learning, Spring 2023 \\ \smallskip
Pratik Patil}
\author{}
\date{}

\usepackage{pratik}
\usepackage{parskip}

\begin{document}
\maketitle
\RaggedRight
\vspace{-50pt}

Note: The content of this supplement is pretty much taken shamelessly 
from Appendix C of \cite{patil_2022}
(continuing the trend from Ryan's supplementary notes on B-splines).

\section{Calculus of asymptotic equivalents}
\label{sec:calculus_deterministic_equivalents}

We will use the language of asymptotic equivalents
to express the limiting bias and variance resolvents of ridge regression.
In this section, we provide a basic review of the definition
of asymptotic equivalents and list several useful calculus rules
that such equivalence obeys.
For more details, see \cite{dobriban_sheng_2021, patil_2022}.
The treatment below borrows heavily from the latter reference.


\begin{definition}
    [Asymptotic equivalence]
    \label{def:deterministic-equivalence}
    Consider sequences $\{ A_p \}_{p \ge 1}$ and $\{ B_p \}_{p \ge 1}$
    of (random or deterministic) matrices of growing dimension.
    We say that $A_p$ and $B_p$ are asymptotically equivalent and write
    $A_p \asympequi B_p$ if
    $\lim_{p \to \infty} | \tr[C_p (A_p - B_p)] | = 0$ almost surely
    for any sequence $C_p$ matrices with bounded trace norm
    such that $\limsup \| C_p \|_{\mathrm{tr}} < \infty$
    as $p \to \infty$.
\end{definition}

An observant reader will notice that
\cite{dobriban_sheng_2021}
use the notation $A_p \asymp B_p$
to denote asymptotic equivalence.
% In this paper, 
We instead prefer to use the notation
$A_p \asympequi B_p$ for such equivalence
to stress the fact that this equivalence
is exact in the limit rather than up to constants
as the ``standard'' use of the asymptotic notation
$\asymp$ would hint at.

\begin{lemma}
    [Calculus of asymptotic equivalents, \cite{dobriban_wager_2018}, \cite{dobriban_sheng_2021}]
    \label{lem:calculus-detequi}
    Let $A_p$, $B_p$, and $C_p$ be sequences of (random or deterministic) matrices.
    The calculus of deterministic equivalents satisfy the following properties:
    \begin{enumerate}
        \item 
        \label{lem:calculus-detequi-item-equivalence}
        Equivalence:
        The relation $\asympequi$ is an equivalence relation.
        \item 
        \label{lem:calculus-detequi-item-sum}
        Sum:
        If $A_p \asympequi B_p$ and $C_p \asympequi D_p$, then $A_p + C_p \asympequi B_p + D_p$.
        \item 
        \label{lem:calculus-detequi-item-product}
        Product:
        If $A_p$ a sequence of matrices with bounded operator norms, i.e., $\| A_p \|_{\op} < \infty$,
        and $B_p \asympequi C_p$, then $A_p B_p \asympequi A_p C_p$.
        \item 
        \label{lem:calculus-detequi-item-trace}
        Trace:
        If $A_p \asympequi B_p$, then $\tr[A_p] / p - \tr[B_p] / p \to 0$ almost surely.
        \item 
        \label{lem:calculus-detequi-item-differentiation}
        Differentiation:
        Suppose $f(z, A_p) \asympequi g(z, B_p)$ where the entries of $f$ and $g$
        are analytic functions in $z \in S$ and $S$ is an open connected subset of $\CC$.
        Suppose for any sequence $C_p$ of deterministic matrices with bounded trace norm
        we have $| \tr[C_p (f(z, A_p) - g(z, B_p))] | \le M$ for every $p$ and $z \in S$.
        Then, we have $f'(z, A_p) \asympequi g'(z, B_p)$ for every $z \in S$,
        where the derivatives are taken entry-wise with respect to $z$.
    \end{enumerate}
\end{lemma}

The notion of \emph{deterministic} equivalence is a special case
of asymptotic equivalence 
where one of the sequences is a deterministic sequence. 
In the sequel, 
we will first record deterministic equivalent for the standard ridge resolvent 
in terms of the population covariance matrix $\Sigma$,
and then derive deterministic equivalents for the bias and variance resolvents
arising in the squared prediction risk of ridge regression.

A side comment:
Some of you may be wondering what the term ``resolvent'' means.
Resolvent formalism is a technique that uses complex-analytic machinery
in the study of the spectrum of operators on Hilbert, Banach spaces, 
and more general spaces.
You can quench your curiosity to know more about this topic at:
\url{https://en.wikipedia.org/wiki/Resolvent_formalism}.

\begin{lemma}
    [Deterministic equivalent for basic ridge resolvent,
    adapted from 
    Theorem 1 of \cite{rubio_mestre_2011};
    see also Theorem 3.1 of \cite{dobriban_sheng_2021}]
    \label{lem:basic-ridge-resolvent-deterministic-equivalent}
    Suppose $x_i \in \RR^{p}$, $1 \le i \le n$, are i.i.d.\ random vectors
    where each $x_i = z_i \Sigma^{1/2}$,
    where $z_i$ contains i.i.d.\ entries $z_{ij}$, $1 \le j \le p$,
    with $\EE[z_{ij}] = 0$, $\EE[z_{ij}^2] = 1$, and $\EE[|z_{ij}|^{8+\alpha}] \le M_\alpha$
    for some $\alpha > 0$ and $M_\alpha < \infty$,
    and $\Sigma \in \RR^{p \times p}$ is a positive semidefinite matrix
    such that $0 \preceq \Sigma \preceq r_{\max} I_p$
    for some constant
    (independent of $p$)
    $r_{\max} < \infty$.
    Let $X \in \RR^{n \times p}$ the matrix with $x_i$, $1 \le i \le n$ as rows
    and $\hSigma \in \RR^{p \times p}$ denote the random matrix $X^\top X / n$.
    Define $\gamma_n = p / n$.
    Then, 
    for $z \in \CC^{> 0}$,
    as $n, p \to \infty$ such that $0 < \liminf \gamma_n \le \limsup \gamma_n < \infty$,
    we have
    \begin{equation}
        (\hSigma - z I_p)^{-1}
        \asympequi
        (c(e(z; \gamma_n)) \Sigma - z I_p)^{-1},
    \end{equation}
    where
    $c(e(z; \gamma_n))$ is defined as
    \begin{equation}
        \label{eq:basic-ridge-equivalence-c-e-relation}
        c(e(z; \gamma_n))
        = \frac{1}{ 1 + \gamma_n e(z; \gamma_n)},
    \end{equation}
    and $e(z; \gamma_n)$ is the unique solution in $\CC^{> 0}$ to the fixed-point equation
    \begin{equation}
        \label{eq:basic-ridge-equivalence-e-fixed-point}
        e(z; \gamma_n)
        = 
        \tr[ \Sigma (c(e(z; \gamma_n)) \Sigma  - z I_p)^{-1} ] / p.
    \end{equation}
\end{lemma}

We note that in defining $e(\lambda; \gamma_n)$,
it is also implicitly a parameterized by $\Sigma$.
We suppress this dependence for notational simplicity,
and only explicitly indicate dependence on 
$z$
and $\gamma_n$
that will be useful for our purposes. 

A helpful corollary of \Cref{lem:basic-ridge-resolvent-deterministic-equivalent}
is the following result that considers the ``scaled'' ridge resolvent.
The reason why such scaling helps is because
in the limit as $\lambda \to 0^+$,
the ridge resolvent itself may blow up,
but the scaled resolvent is well-behaved.


\begin{corollary}
    [Deterministic equivalent for scaled ridge resolvent]
    \label{cor:basic-ridge-resolvent-equivalent-in-v}
    Assume the setting of 
    \Cref{lem:basic-ridge-resolvent-deterministic-equivalent}.
    Then, for $\lambda > 0$,
    as $n, p \to \infty$
    such that $0 < \liminf \gamma_n \le \limsup \gamma_n < \infty$,
    we have
    \[
        \lambda (\hSigma + \lambda I_p)^{-1}
        \asympequi
        (v(-\lambda; \gamma_n) \Sigma + I_p)^{-1},
    \]
    where $v(-\lambda; \gamma_n)$
    is the unique solution to the fixed-point equation
    \[
        \frac{1}{v(-\lambda; \gamma_n)}
        =
        \lambda
        + \gamma_n \tr[\Sigma (v(-\lambda; \gamma_n) \Sigma + I_p)^{-1}] / p.
    \]
\end{corollary}

We remark that in moving from 
\Cref{lem:basic-ridge-resolvent-deterministic-equivalent}
to
\Cref{cor:basic-ridge-resolvent-equivalent-in-v},
we have switched from a complex $z$ to a real $\lambda$.
For more details of how this is done,
see the proof of \Cref{cor:basic-ridge-resolvent-equivalent-in-v}
in Appendix C of \cite{patil_2022}
or the proof of Theorem 5 in \cite{hastie2022surprises}
(that uses
Lemma 2.2 of
\cite{knowles_yin_2017}).

\section{Deterministic equivalents for ridge}

\begin{lemma}
    [Deterministic equivalents for bias and variance ridge resolvents]
    \label{lem:deter-approx-generalized-ridge}
    Assume the setting of \Cref{lem:basic-ridge-resolvent-deterministic-equivalent}.
    Then, for $\lambda > 0$,
    as $n, p \to \infty$
    with $0 < \liminf \gamma_n \le \limsup \gamma_n < \infty$,
    the following asymptotic deterministic equivalences hold:
    \begin{enumerate}
        \item Variance resolvent of ridge regression:
        \begin{equation}
            \label{eq:detequi-ridge-genbias}
            (\hSigma + \lambda I_p)^{-2} \hSigma \Sigma
            \asympequi
            \tv(-\lambda; \gamma_n) (v(-\lambda; \gamma_n) \Sigma + I_p)^{-2} \Sigma^2,
        \end{equation}
        where $v(-\lambda; \gamma_n) \ge 0$
        is the unique solution to the fixed-point equation
        \begin{equation}
            \label{eq:fixed-point-v-ridge-statement}
            v(-\lambda; \gamma_n)^{-1}
            = \lambda + \gamma_n \tr[\Sigma (v(-\lambda; \gamma_n) \Sigma + I_p)^{-1}] / p,
        \end{equation}
        and $\tv(-\lambda; \gamma_n)$ is defined via 
        $v(-\lambda; \gamma_n)$ by the equation
        \begin{equation}
            \label{eq:def-v'-ridge}
            \tv(-\lambda; \gamma_n)^{-1}
            = 
                v(-\lambda; \gamma_n)^{-2}
                - 
                \gamma_n
                \tr[\Sigma^2 (v(-\lambda; \gamma_n) \Sigma + I_p)^{-2}] / p.
        \end{equation}
        \item Bias resolvent of ridge regression:
        \begin{equation}
            \label{eq:detequi-ridge-genvar}
            \lambda^2
            (\hSigma + \lambda I_p)^{-1} \Sigma (\hSigma + \lambda I_p)^{-1}
            \asympequi 
            (1 + \tv_b(-\lambda; \gamma_n))
            (v(-\lambda; \gamma_n) \Sigma + I_p)^{-1}
            % (\tv_b(-\lambda; \gamma_n) \Sigma + A)
            \Sigma
            (v(-\lambda; \gamma_n) \Sigma + I_p)^{-1},
        \end{equation}
        where $v(-\lambda; \gamma_n)$ as defined in \eqref{eq:fixed-point-v-ridge},
        and $\tv_b(-\lambda; \gamma_n)$
        is defined via $v(-\lambda; \gamma_n)$ by the equation
        \begin{equation}
            \label{eq:def-vg'-ridge}
            \tv_b(-\lambda; \gamma_n)
            =
            \ddfrac
            {
                \gamma_n \tr[\Sigma^2 (v(-\lambda; \gamma_n) \Sigma + I_p)^{-2}] / p
            }
            {
                v(-\lambda; \gamma_n)^{-2}
                - \gamma_n \tr[\Sigma^2 (v(-\lambda; \gamma_n) \Sigma + I_p)^{-2}] / p
            }.
        \end{equation}
    \end{enumerate}
\end{lemma}
\begin{proof}
    The plan of attack for both the first and second parts is to use 
    \Cref{cor:basic-ridge-resolvent-equivalent-in-v}
    as the starting point,
    and apply the calculus rules for asymptotic deterministic equivalents 
    listed in \Cref{sec:calculus_deterministic_equivalents}
    to manipulate into the desired equivalents.
    
    \paragraph{Part 1.}
    For the first part,
    observe that we can express the resolvent of interest 
    (associated with the variance of ridge regression)
    as a derivative (with respect to $\lambda$) of a certain resolvent:
   \begin{equation}
        \label{eq:deriv-def-genvar-ridge}
        (\hSigma + \lambda I_p)^{-2} \hSigma \Sigma
        = (\hSigma + \lambda I_p)^{-1} \Sigma - \lambda (\hSigma + \lambda I_p)^{-2} \Sigma
        = 
        \frac{\partial}{\partial \lambda}
        [\lambda (\hSigma + \lambda I_p)^{-1} \Sigma)].
   \end{equation}
   To find a deterministic equivalent for $(\hSigma + \lambda I_p)^{-2} \hSigma \Sigma$,
   it thus suffices to obtain a deterministic equivalent for the resolvent
   $\lambda (\hSigma + \lambda I_p)^{-1} \Sigma$
   and take its derivative, 
   thanks to the differentiation rule from
    \Cref{lem:calculus-detequi}~\eqref{lem:calculus-detequi-item-differentiation}.
    
    Starting with
    \Cref{cor:basic-ridge-resolvent-equivalent-in-v},
    we have
    \[
        \lambda (\hSigma + \lambda I_p)^{-1}
        \asympequi (v(-\lambda; \gamma_n) \Sigma + I_p)^{-1},
    \]
  where $v(-\lambda; \gamma_n)$ is the unique solution to the fixed point equation
  \begin{equation}
        \label{eq:fixed-point-v-ridge}
        v(-\lambda; \gamma_n)^{-1}
        = \lambda + \gamma_n \tr[\Sigma (v(-\lambda; \gamma_n) \Sigma + I_p)^{-1}] / p.
  \end{equation}
   Since $\Sigma$ has bounded operator norm (uniformly in $p$),
   from 
    \Cref{lem:calculus-detequi}~\eqref{lem:calculus-detequi-item-product},
   we have
  \begin{equation}
        \label{eq:variance-base-functional-equivalence-ridge}
        \lambda (\hSigma + \lambda I_p)^{-1} \Sigma
        \asympequi (v(-\lambda; \gamma_n) \Sigma + I_p)^{-1} \Sigma,
  \end{equation}
  where $v(-\lambda; \gamma_n)$ is as defined by \eqref{eq:fixed-point-v-ridge}.
    It now remains to take the derivative of 
    the right hand side of \eqref{eq:variance-base-functional-equivalence-ridge}
    with respect to $\lambda$.
    One can verify that the differentiation rule indeed applies in this case.
    (You should check this!)
    Taking derivative,
    we have
    \begin{equation}
        \label{eq:deriv-detequi-genvar-ridge}
        \frac{\partial}{\partial \lambda}
        [(v(-\lambda; \gamma_n) \Sigma + I_p)^{-1} \Sigma]
        = - \frac{\partial}{\partial \lambda}[v(-\lambda; \gamma_n)]
        (v(-\lambda; \gamma_n) \Sigma + I_p)^{-2} \Sigma^2.
    \end{equation}
    We can write 
    - $\partial / \partial \lambda [v(-\lambda; \gamma_n)]$
    in terms of $v(-\lambda; \gamma_n)$
    by taking derivative of \eqref{eq:fixed-point-v-ridge}
    with respect to $\lambda$
    and solving for 
    - $\partial / \partial \lambda [v(-\lambda; \gamma_n)]$.
    Taking the derivative of \eqref{eq:fixed-point-v-ridge} yields
    the following equation:
    \begin{equation}
        - \label{eq:deriv-relation-v-ridge}
        \frac{\partial}{\partial \lambda} [v(-\lambda; \gamma_n)]
        v(-\lambda; \gamma_n)^{-2}
        =  1 + \gamma_n - \frac{\partial}{\partial \lambda}[v(-\lambda; \gamma_n)] 
        \tr[\Sigma^2 (v(-\lambda; \gamma_n) \Sigma + I_p)^{-2}] / p.
    \end{equation}
    Denoting 
    - $\partial / \partial \lambda [v(-\lambda; \gamma_n)]$
    by $\tv(-\lambda; \gamma_n)$
    and solving for $\tv(-\lambda; \gamma_n)$
    in \eqref{eq:deriv-relation-v-ridge},
    we get
    \begin{equation}
        \label{eq:def-tv-ridge-in-the-proof}
        \tv(-\lambda; \gamma_n)^{-1}
        = 
        v(-\lambda; \gamma_n)^{-2}
        - \gamma_n \tr[\Sigma^2 (v(-\lambda; \gamma_n) \Sigma + I_p)^{-2}] / p.
    \end{equation}
    Combining 
    \eqref{eq:deriv-def-genvar-ridge},
    \eqref{eq:deriv-detequi-genvar-ridge},
    and 
    \eqref{eq:def-tv-ridge-in-the-proof},
    the statement follows.
    This completes the proof of the first part.
    
    
    \paragraph{Part 2.}
    For the second part,
    observe that we can express
    the resolvent of interest 
    (appearing in the bias of ridge regression)
    as a derivative 
    of a certain parameterized resolvent at a fixed value
    of the parameter:
    \begin{equation}
        \label{eq:deriv-def-genbias-ridge}
        \begin{split}
        \lambda^2 
        (\hSigma + \lambda I_p)^{-1}
        \Sigma
        (\hSigma + \lambda I_p)^{-1}
        &=
        \lambda^2
        (\hSigma + \lambda I_p + \lambda \rho \Sigma)^{-1} 
        \Sigma 
        (\hSigma + \lambda I_p + \lambda \rho \Sigma)^{-1} |_{\rho = 0} \\
        &=
        -
        \frac{\partial}{\partial \rho}
        [\lambda (\hSigma + \lambda I_p + \lambda \rho \Sigma)^{-1}] \mathrel{\Big |}_{\rho = 0}.
        \end{split}
    \end{equation}
    It is worth remarking that in contrast to Part 1, we needed
    to introduce another parameter $\rho$ for this part to appropriately
    pull out the matrix $\Sigma$ in the middle.
    This trick has been used in the proof
    of Theorem 5 in \cite{hastie2022surprises}
    in the context of bias calculation for ridge regression.
    Our strategy henceforth will be to
    obtain a deterministic equivalent
    for the resolvent $\lambda (\hSigma + \lambda I_p + \lambda \rho \Sigma)^{-1}$,
    take its derivative with respect to $\rho$, and set $\rho = 0$.
    Towards that end,
    we first massage it
    to make it amenable
    for application of \Cref{lem:basic-ridge-resolvent-deterministic-equivalent}
    as follows:
   \begin{align}
        \lambda
        \big(\hSigma + \lambda I_p + \lambda \rho \Sigma\big)^{-1}
        &=
        \lambda
        \big(\hSigma + \lambda (I_p + \rho \Sigma)\big)^{-1} 
        \nonumber \\
        &=
        (I_p + \rho \Sigma)^{-1/2}
        \lambda\big((I_p + \rho \Sigma)^{-1/2} \hSigma (I_p + \rho \Sigma)^{-1/2} + \lambda I_p\big)^{-1}
        (I_p + \rho \Sigma)^{-1/2} 
        \nonumber \\
        &=
        (I_p + \rho \Sigma)^{-1/2}
        \lambda
        \big(
        \hSigma_{\rho, \Sigma} + \lambda I_p
        \big)^{-1}
        (I_p + \rho \Sigma)^{-1/2}, \label{eq:resolvent-massage-genbias-ridge}
   \end{align}
   where $\hSigma_{\rho, \Sigma} := \Sigma_{\rho, \Sigma}^{1/2} (Z^\top Z / n) \Sigma_{\rho, \Sigma}^{1/2}$
   and $\Sigma_{\rho, \Sigma} := (I_p + \rho \Sigma)^{-1/2} \Sigma (I_p + \rho \Sigma)^{-1/2}$.
   We will now obtain a deterministic equivalent
   for $\lambda (\hSigma_{\rho, \Sigma} + \lambda I_p)^{-1}$,
   and use the product rule to arrive at the deterministic
   equivalent for $\lambda (\hSigma + \lambda I_p + \lambda \rho \Sigma)^{-1}$.
   
   Using 
   \Cref{cor:basic-ridge-resolvent-equivalent-in-v},
   we have
   \begin{equation}
        \label{eq:resolvent-in-v-genbias-ridge}
        \lambda (\hSigma_{\rho, \Sigma} + \lambda I_p)^{-1}
        \asympequi (v_b(-\lambda, \rho; \gamma_n) \Sigma_{\rho, \Sigma} + I_p)^{-1},
   \end{equation}
   where $v_b(-\lambda, \rho; \gamma_n)$
   is the unique solution to the fixed-point equation
   \begin{equation}
        \label{eq:fixed-point-in-v-genbias-ridge}
        v_b(-\lambda, \rho; \gamma_n)^{-1}
        = \lambda + \gamma_n 
        \tr[\Sigma_{\rho, \Sigma}  (v_b(-\lambda, \rho; \gamma_n) \Sigma_{\rho, \Sigma} + I_p)^{-1}] / p.
   \end{equation}
   Combining 
   \eqref{eq:resolvent-massage-genbias-ridge}
   with 
   \eqref{eq:resolvent-in-v-genbias-ridge},
   and using the product rule from 
    \Cref{lem:calculus-detequi}~\eqref{lem:calculus-detequi-item-product}
   (which is applicable since $(I_p + \rho \Sigma)^{-1/2}$ is a deterministic matrix),
   we get
   \begin{align*}
        \lambda (\hSigma + \lambda I_p + \lambda \rho \Sigma)^{-1}
        &= 
        (I_p + \rho \Sigma)^{-1/2}
        \lambda
        (\hSigma_{\rho, \Sigma} + \lambda I_p)^{-1}
        (I_p + \rho \Sigma)^{-1/2}
        \\
        &\asympequi
        (I_p + \rho \Sigma)^{-1/2}
        (v_b(-\lambda, \rho; \gamma_n) \Sigma_{\rho, \Sigma} + I_p)^{-1}
        (I_p + \rho \Sigma)^{-1/2} \\
        &= 
        (I_p + \rho \Sigma)^{-1/2}
        (v_b(-\lambda, \rho; \gamma_n) (I_p + \rho \Sigma)^{-1/2} \Sigma (I_p + \rho \Sigma)^{-1/2}  + I_p)^{-1}
        (I_p + \rho \Sigma)^{-1/2} \\
        &=
        (v_b(-\lambda, \rho; \gamma_n) \Sigma + I_p + \rho \Sigma)^{-1}.
   \end{align*}
   Similarly, 
   the right hand side of the fixed-point equation 
   \eqref{eq:fixed-point-in-v-genbias-ridge} 
   can be simplified
   by substituting back for $\Sigma_{\rho, \Sigma}$ to yield
   \begin{align}
       v_b(-\lambda, \rho; \gamma_n)^{-1}
       &= \lambda + \gamma_n 
       \tr[(I_p + \rho \Sigma)^{-1/2} \Sigma (I_p + \rho \Sigma)^{-1/2} 
       (v_b(-\lambda, \rho; \gamma_n) \Sigma_{\rho, \Sigma} + I_p)^{-1}] / p \nonumber \\
       &=
       \lambda + \gamma_n
       \tr[ \Sigma 
       (v_b(-\lambda, \rho; \gamma_n) 
       (I_p + \rho \Sigma)^{1/2} \Sigma_{\rho, \Sigma} (I_p + \rho \Sigma)^{1/2} 
       + (I_p + \rho \Sigma))^{-1}] / p \nonumber \\
       &=
       \lambda + \gamma_n
       \tr[\Sigma (v_b(-\lambda, \rho; \gamma_n) \Sigma + I_p + \rho \Sigma)^{-1}] / p. \label{eq:fixed-point-vg-with-rho}
   \end{align}
   Finally, we will now use
  the differentiation rule 
    from \Cref{lem:calculus-detequi}~\eqref{lem:calculus-detequi-item-differentiation}
   (with respect to $\rho$ this time).
   It is easy verify that the differentiation rule applies
   in the neighborhood of $\rho = 0$.
    (Again, check this!)
    Taking derivative with respect to $\rho$,
    we get
   \begin{equation}
        \label{eq:deriv-detequi-genbias-ridge}
        \begin{split}
        &-
        \frac{\partial}{\partial \rho}
        [ ( v_b(-\lambda, \rho; \gamma_n) \Sigma + I_p + \rho \Sigma )^{-1} ] \\
        &\quad = (v_b(-\lambda, \rho; \gamma_n) \Sigma + I_p + \rho \Sigma)^{-1}
        \left(\frac{\partial}{\partial \rho}[v_b(-\lambda, \rho; \gamma_n)] \Sigma + \Sigma\right)
        (v_b(-\lambda, \rho; \gamma_n) \Sigma + I_p + \rho \Sigma)^{-1}.
        \end{split}
   \end{equation}
   Setting $\rho = 0$ and observing that
   $v_b(-\lambda, 0; \gamma_n) = v(-\lambda; \gamma_n)$,
   where $v(-\lambda; \gamma_n)$ is as defined in \eqref{eq:fixed-point-v-ridge},
    we have
    \begin{equation}
        \label{eq:deriv-detequi-genbias-ridge-rho0}
        \begin{split}
        &\frac{\partial}{\partial \rho}
        [
        (v_b(-\lambda, \rho; \gamma_n) \Sigma + I_p + \rho \Sigma)^{-1}
        ] \mathrel{\Big |}_{\rho = 0} \\
        &\quad = 
        (v(-\lambda; \gamma_n) \Sigma + I_p)^{-1}
        \left( 
            \frac{\partial}{\partial \rho}[v_b(-\lambda, \rho; \gamma_n)] 
            \mathrel{\Big|}_{\rho = 0} 
            \Sigma  + \Sigma 
        \right)
        (v(-\lambda; \gamma_n) \Sigma + I_p)^{-1}.
        \end{split}
    \end{equation}
    To obtain an equation for 
    $\partial / \partial \rho [v_b(-\lambda, \rho; \gamma_n)] |_{\rho = 0}$,
    we can differentiate the fixed-point equation \eqref{eq:fixed-point-vg-with-rho}
    with respect to $\rho$ to yield
    % \begin{multline}
    \begin{equation*}
    \begin{split}
        &-
        \frac{\partial}{\partial \rho}
        [v_b(-\lambda, \rho; \gamma_n)] 
        v_b(-\lambda, \rho; \gamma_n)^{-2} \\
        &\quad = 
        - \gamma_n 
        \frac{\partial}{\partial \rho}[v_b(-\lambda, \rho; \gamma_n)]
        \tr[\Sigma^2 (v_b(-\lambda, \rho; \gamma_n) \Sigma + I_p + \rho \Sigma)^{-2}] / p \\
        &\qquad \qquad - \gamma_n
        \tr[\Sigma^2 (v_b(-\lambda, \rho; \gamma_n) \Sigma + I_p + \rho \Sigma)^{-2}] / p.
    \end{split}
    \end{equation*}
    % \end{multline}
    Setting $\rho = 0$ in the equation above,
    and using the fact that $v_b(-\lambda, 0; \gamma_n) = v(-\lambda; \gamma_n)$,
    and denoting 
    $\partial / \partial \rho[v_b(-\lambda, \rho; \gamma_n)] |_{\rho = 0}$
    by $\tv_b(-\lambda; \gamma_n)$,
    we get that
    \begin{equation}
        \label{eq:def-tvg-ridge-in-the-proof}
        \tv_b(-\lambda; \gamma_n)
        = 
        \ddfrac
        {\gamma_n \tr[\Sigma^2 (v(-\lambda; \gamma_n) \Sigma + I_p)^{-2}] / p}
        {v(-\lambda; \gamma_n)^{-2} - \gamma_n \tr[\Sigma^2 (v(-\lambda; \gamma_n) \Sigma + I_p)^{-2}] / p}.
    \end{equation}
    Therefore, from \eqref{eq:deriv-def-genbias-ridge} 
    and \eqref{eq:deriv-detequi-genbias-ridge-rho0},
    we finally have
   \begin{align*}
        \lambda^2
        (\hSigma + \lambda I_p)^{-1}
        \Sigma
        (\hSigma + \lambda I_p)^{-1}
        &\asympequi
        (v(-\lambda; \gamma_n) \Sigma + I_p)^{-1}
        (\tv_b(-\lambda; \gamma_n) \Sigma + \Sigma)
        (v(-\lambda; \gamma_n) \Sigma + I_p)^{-1} \\
        &=
        (1 + \tv_b(-\lambda; \gamma_n))
        (v(-\lambda; \gamma_n) \Sigma + I_p)^{-1}
        \Sigma
        (v(-\lambda; \gamma_n) \Sigma + I_p)^{-1},
   \end{align*}
   where $v(-\lambda; \gamma_n)$ is as defined in \eqref{eq:fixed-point-v-ridge},
   and $\tv_b(-\lambda; \gamma_n)$ is as defined in \eqref{eq:def-tvg-ridge-in-the-proof}.
   This completes the proof of the second part.
\end{proof}

\section{Deterministic equivalents for ridgeless}

\begin{lemma}
    [Deterministic equivalents for bias and variance ridgeless resolvents]
    \label{lem:deter-approx-generalized-ridgeless}
    Assume the setting of \Cref{lem:deter-approx-generalized-ridge}
    with $\gamma_n \in (1, \infty)$.
    Then,
    the following deterministic equivalences hold:
    \begin{enumerate}
        \item Variance resolvent of ridgeless regression:
        \begin{equation}
            \label{eq:detequi-mn2ls-genbias}
            \hSigma^{+} \Sigma
            \asympequi
            \tv(0; \gamma_n)
            (v(0; \gamma_n) \Sigma + I_p)^{-2} \Sigma^2,
        \end{equation}
        where $v(0; \gamma_n)$ is the unique solution
        to the fixed-point equation
        \begin{equation}
            \label{eq:fixed-point-v-mn2ls}
            \gamma_n^{-1}
            = \tr[v(0; \gamma_n) \Sigma (v(0; \gamma_n) \Sigma + I_p)^{-1}] / p,
        \end{equation}
        and $\tv(0; \gamma_n)$ is defined through $v(0; \gamma_n)$  via
        \begin{equation}
            \label{eq:def-vg-mn2ls-tv}
            \tv(0; \gamma_n)
            = \big( v(0; \gamma_n)^{-2} - \gamma_n \tr[\Sigma^2 (v(0; \gamma_n) \Sigma + I_p)^{-2}] / p \big)^{-1}.
        \end{equation}
        \item Bias resolvent of ridgeless regression:
        \begin{equation}
            (I_p - \hSigma^{+} \hSigma)
            \Sigma
            (I_p - \hSigma^{+} \hSigma)
            \asympequi
            (1 + \tv_b(0; \gamma_n))
            (v(0; \gamma_n) \Sigma + I_p)^{-1}
            \Sigma
            (v(0; \gamma_n) \Sigma + I_p)^{-1},
        \end{equation}
    \end{enumerate}
\end{lemma}
where $v(0; \gamma_n)$ is as defined in \eqref{eq:fixed-point-v-mn2ls},
and $\tv_b(0; \gamma_n)$ is defined via $v(0; \gamma_n)$ by
\begin{equation}
    \label{eq:def-tv-mn2ls-tvg}
    \tv_b(0; \gamma_n)
    =
    \gamma_n
    \tr[\Sigma^2 (v(0; \gamma_n) \Sigma + I_p)^{-2}] / p
    \cdot
    \big(v(0; \gamma_n)^{-2} - \gamma_n \tr[\Sigma^2 (v(0; \gamma_n) \Sigma + I_p)^{-2}] / p\big)^{-1}.
\end{equation}

\begin{proof}
    The plan of attack 
    for both the parts is to use the results of \Cref{lem:deter-approx-generalized-ridge}
    and limiting arguments as $\lambda \to 0^{+}$.
    The results of \Cref{lem:deter-approx-generalized-ridge}
    are pointwise in $\lambda$,
    but can be strengthened to be uniform in $\lambda$
    over a range
    that includes $\lambda = 0$
    allowing one to take the limits of the deterministic equivalents
    obtained in \Cref{lem:deter-approx-generalized-ridge}
    as $\lambda \to 0^{+}$.
    
    \paragraph{Part 1.}
    We will use the result in Part 1 of \Cref{lem:deter-approx-generalized-ridge}
    as our starting point.
    Let $\Lambda := [0, \lambda_{\max}]$
    where $\lambda_{\max} < \infty$,
    and let $T$ be a matrix with bounded trace norm.
    Note that
    \begin{equation}
        \label{eq:var-func-bound}
        \vert \tr[(\hSigma + \lambda I_p)^{-2} \hSigma \Sigma T ] \vert
        \le \| (\hSigma + \lambda I_p)^{-2} \hSigma \Sigma \|_{\mathrm{op}} \tr[T]
        \le C \| (\hSigma + \lambda I_p)^{-2} \hSigma \|_{\mathrm{op}} 
        \| \Sigma \|_{\mathrm{op}}
        \le C
    \end{equation}
    for some constant $C < \infty$.
    Here, the last inequality follows
    because $s_i^2 / (s_i^2 + \lambda)^2 \le 1$
    where $s_i^2$, $1 \le i \le p$,
    are the eigenvalues of $\hSigma$,
    and the operator norm $\Sigma$ is assumed to be bounded.
    Consider the magnitude of the derivative (in $\lambda$) of the map
     $\lambda
     \mapsto
     \tr[(\hSigma + \lambda I_p)^{-2} \hSigma \Sigma T]$
     given by
    \[
        \left|
        \frac{\partial}{\partial \lambda}
        \tr[(\hSigma + \lambda I_p)^{-2} \hSigma \Sigma T]
        \right|
        = 2 \vert \tr[(\hSigma + \lambda I_p)^{-3} \hSigma \Sigma T] \vert.
    \]
    Following the argument in \eqref{eq:var-func-bound},
    for $\lambda \in \Lambda$,
    observe that
    \[
        \vert \tr[(\hSigma + \lambda I_p)^{-3} \hSigma \Sigma T] \vert
        \le \| (\hSigma + \lambda I_p)^{-3} \hSigma \|_{\mathrm{op}} 
        \| \Sigma \|_{\mathrm{op}} \tr[T]
        \le C
    \]
    for some constant $C < \infty$.
    Similarly,
    in the same interval
    $
        \tr
        [
        \tv(-\lambda; \gamma_n)
        (v(-\lambda; \gamma_n) \Sigma + I_p)^{-2}
        \Sigma^2 T
        ]
        \le C.
    $
    In addition,
    it is easy to show that
    the map
    $
        \lambda 
        \mapsto
        \tr[\tv(-\lambda; \gamma_n) 
        (v(-\lambda; \gamma_n) \Sigma + I_p)^{-2} \Sigma T]
    $
    is differentiable in $\lambda$
    and that 
    the derivative for $\lambda \in \Lambda$ is bounded.
    Therefore,
    the family of functions
    $\tr[ (\hSigma + \lambda I_p)^{-2} \hSigma \Sigma T] 
    - \tr[\tv(-\lambda; \gamma_n) (v(-\lambda; \gamma_n) \Sigma + I_p)^{-2} \Sigma^2 T]$
    forms an equicontinuous family in $\lambda$
    over $\lambda \in \Lambda$.
    Thus, the convergence in Part 1 of \Cref{lem:deter-approx-generalized-ridge} 
    is uniform in $\lambda$.
    We can now use the Moore-Osgood theorem
    to interchange the limits
    to obtain
    \begin{align*}
        &\lim_{p \to \infty}
        \Big\{ \tr[\hSigma^{+} \Sigma T]
        - \tr[\tv(0; \gamma_n) (v(0; \gamma_n) \Sigma + I_p)^{-2} \Sigma^2 T] \Big\} \\
        &=
        \lim_{p \to \infty}
        \lim_{\lambda \to 0^{+}}
        \Big\{ \tr[(\hSigma + \lambda I_p)^{-2} \hSigma \Sigma T]
        - \tr[\tv(-\lambda; \gamma_n) (v(-\lambda; \gamma_n) \Sigma + I_p)^{-2} \Sigma^2 T)] \Big\} \\
        &=
        \lim_{\lambda \to 0^{+}}
        \lim_{p \to \infty}
        \Big\{
        \tr[(\hSigma + \lambda I_p)^{-2} \hSigma \Sigma T]
        - \tr[\tv(-\lambda; \gamma_n) (v(-\lambda; \gamma_n) \Sigma + I_p)^{-2} \Sigma^2 T)] \Big\} \\
        &= 0.
    \end{align*}
    In the first equality above,
    we used the fact that 
    $\hSigma^{+} = \hSigma^{+} \hSigma \hSigma^{+} 
    = \lim_{\lambda \to 0^{+}} (\hSigma + \lambda I_p)^{-1} \hSigma (\hSigma + \lambda I_p)^{-1}$,
    and that the functions
    $v(\cdot; \gamma_n)$
    and
    $\tv(\cdot; \gamma_n)$
    are continuous
    (which is easy to verify). 
    This
    provides the right hand side of \eqref{eq:detequi-mn2ls-genbias}.
    Similarly, the fixed-point equation \eqref{eq:fixed-point-v-ridge}
    as $\lambda \to 0^{+}$ becomes
    \[
        v(0; \gamma_n)^{-1}
        = \gamma_n \tr[\Sigma (v(0; \gamma_n) \Sigma + I_p)^{-1}] / p.
    \]
    Moving $v(0; \gamma_n)$ to the other side
    (it follows easily that $v(0; \gamma_n) > 0$
    for $\gamma_n \in (1, \infty)$, so we are safe in doing this),
    we arrive at the desired result.
    
    \paragraph{Part 2.}
    As done in Part 1,
    it is not difficult to show
    that over $\lambda \in \Lambda$
    the family of functions
    $\tr[\lambda^2 (\hSigma + \lambda I_p)^{-1} \Sigma (\hSigma + \lambda I_p)^{-1} T]
    - \tr[(1 + \tv_b(-\lambda; \gamma_n)) (v(-\lambda; \gamma_n) \Sigma + I_p)^{-1} \Sigma 
    (v(-\lambda; \gamma_n) \Sigma + I_p)^{-1} T]$
    form an equicontinuous family.
    Therefore,
    the convergence in Part 2 of 
    \Cref{lem:deter-approx-generalized-ridge}
    is uniform in $\lambda$ over $\Lambda$ (that includes $0$).
    Using the Moore-Osgood theorem
    to the interchange the limits,
    one has
    \begin{align*}
        &\lim_{p \to \infty}
        \Big\{
        \tr[(I_p - \hSigma^{+} \hSigma) \Sigma (I_p - \hSigma^{+} \hSigma) T] \\
        &\qquad \qquad
        -
        \tr[ 
        (1 + \tv_b(0; \gamma_n))
        (v(0; \gamma_n) \Sigma + I_p)^{-1} 
        % (\tv_b(0; \gamma_n) \Sigma + \Sigma) 
        \Sigma
        (v(0; \gamma_n) \Sigma + I_p)^{-1}
        T
        ]  \Big\}
        \\
        &=
        \lim_{p \to \infty}
        \lim_{\lambda \to 0^{+}}
        \Big\{
        \tr[\lambda^2 (\hSigma + \lambda I_p)^{-1} \Sigma (\hSigma + \lambda I_p)^{-1} T] \\
        &\qquad \qquad \qquad \qquad - \tr[ (1 + \tv_b(-\lambda; \gamma_n))  (v(-\lambda; \gamma_n) \Sigma + I_p)^{-1} 
        % (\tv_b(-\lambda; \gamma_n) \Sigma + \Sigma) 
        \Sigma
        (v(-\lambda; \gamma_n) \Sigma + I_p)^{-1} T] \Big\} \\
        &=
        \lim_{\lambda \to 0^{+}}
        \lim_{p \to \infty}
        \Big\{
        \tr[\lambda^2 (\hSigma + \lambda I_p)^{-1} \Sigma (\hSigma + \lambda I_p)^{-1} T] \\
        &\qquad \qquad \qquad \qquad - \tr[ (1 + \tv_b(-\lambda; \gamma_n)) (v(-\lambda; \gamma_n) \Sigma + I_p)^{-1}
        % (\tv_b(-\lambda; \gamma_n) \Sigma + \Sigma) 
        \Sigma
        (v(-\lambda; \gamma_n) \Sigma + I_p)^{-1} T] \Big\} \\
        &= 0.
    \end{align*}
    Now, both \eqref{eq:def-vg-mn2ls-tv} and \eqref{eq:def-tv-mn2ls-tvg}
    follow by taking $\lambda \to 0^{+}$
    in 
    \eqref{eq:detequi-ridge-genvar}
    and
    \eqref{eq:def-vg'-ridge},
    respectively.
\end{proof}

\bibliographystyle{plainnat}
\bibliography{pratik.bib}

\end{document}